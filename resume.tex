%%%%%%%%%%%%%%%%%%%%%%%%%%%%%%%%%%%%%%%%%%%%%%%%%%%%%%%%%%%%%%%%%%%%%%%%%%%%%%%%
%% Andrew Choens - Resume
%%
%% Based on Michael DeCorte's res.cls.
%%%%%%%%%%%%%%%%%%%%%%%%%%%%%%%%%%%%%%%%%%%%%%%%%%%%%%%%%%%%%%%%%%%%%%%%%%%%%%%%

\documentclass[line, mm, 10pt]{res}
\usepackage[dvipsnames]{xcolor}
\usepackage[english]{babel}
\usepackage{hyperref}
\usepackage[margin=1.0in]{geometry}

\hypersetup{
  colorlinks=true,
  linkcolor=Blue,
  linkbordercolor=Blue,
  pdfborderstyle={/S/U/W 1}
}

\begin{document}

\name{Andrew Choens, MSW - \Resume}
\address{\href{mailto:andy.choens@gmail.com}{andy.choens@gmail.com}}
\address{\href{tel:(518) 275-5984}{(518) 275-5984}}

\begin{resume}

%  \vspace{.01in}
%  \begin{tabular} {c | c | c}
%    {\bf Policy Analyst} & {\bf Programmer} & {\bf Social Worker}
%  \end{tabular}
%  \vspace{.01in}

  \section{Analytics/DATA SCIENCE EXPERIENCE:}

  %% Employer Template ---------------------------------------------------------
  %\employer{\href{URL}{NAME}}
  %\title{TitleName}
  %\dates{Dates01 - Dates02}
  %\location{Albany, NY}
  %\begin{position}
  %  PlaceholderTextIsHere
  %\end{position}

  %% ACPHS -------------------------------------------------------------------
  \employer{\href{https://www.acphs.edu/}{Albany College of Pharmacy and Health Sciences}}
  \title{Adjunct Professor}
  \dates{Spring 2018 - Present}
  \location{Albany, NY}
  \begin{position}
    I teach two undergraduate data classes. These classes teach practical
    analytical skills, data management, and R programming. Both classes include
    readings and discussions "Beyond the Numbers" which include topics such as
    bias in data/analytics, ethics in analytics/machine learning, HIPAA, etc.
    \newline
    \begin{itemize}
      \item[1.] Intro To Data with R introduces students to the R programming
      language. The class builds on the knowledge gained in their statistics
      course and introduces core R programming skills such as data management,
      data clean up, single table data transformations as well as linear and
      logistic regression.
      \item[2.] Applied Public Health Data with R builds on the previous course. This course introduces basic relational data modeling, more complex data management and introduces students to some basic machine learning such as decision trees and random forests.
    \end{itemize}

  \end{position}
  
  %% NTYSTEC ------------------------------------------------------------
  \employer{\href{https://www.nystec.com}{NYSTEC}}
  \title{Consultant/Analyst}
  \dates{January 2023 - Present}
  \location{Albany, NY}
  \begin{position}
    The New York State Technology Enterprise Corporation is a technology
    consulting firm which provides services and guidance for the Federal and
    state governments, as well as local government agencies in New England.

    I provide support services for the Medicaid System Innovation (MSI) team
    which provides services for New York State Medicaid program. In addition to
    providing direct analytical and workflow services to the client, I help the
    MSI team improve internal processes and outcomes. I also mentor junior staff
    and provide support on new services which existing staff may not have
    experience with such as NLP, machine learning, and AI.

    Data sources include NYS Medicaid, NYS Uniform Assessment System (UAS), and
    in-house data sets.
  \end{position}

  %% Charlie Health ------------------------------------------------------------
  \employer{\href{https://www.charliehealth.com}{Charlie Health}}
  \title{Engineering Manager, Data}
  \dates{July 2022 - Present}
  \location{Remote, USA}
  \begin{position}
    Charlie Health is a mental health provider which delivers
    remote-first Intensive Outpatient Care (IOC) to youth and young adults. 
    
    I supervised the nascent data engineering team. We supported business
    operations by improving and democratizing access to Charlie Health's data.
    We implemented a process to slowly transition the operations team from a
    workflow driven by Google Sheets, to a more automated data flow using
    Snowflake and BI tools such as Tableau. By providing automated access to the
    data and reporting, we gave the operations team the time they needed to
    actually lead the operation of the company. 
  \end{position}

  %% Acuitas -------------------------------------------------------------------
  \employer{\href{http://www.acuitashealth.com/}{Acuitas Health}}
  \title{Data Science and Engineering Leader}
  \dates{September 2017 - June 2022}
  \location{Troy, NY}
  \begin{position}
    Acuitas was a population health services organization which helped primary
    care providers succeed in value based contracts with health insurance
    companies. As the leader of the Department of Data Science and Engineering I
    lead the implementation of data-driven services and products. The team I led:
    
    \begin{itemize}
    \item developed a set of process and procedures for the development of high
    quality, validated, data deliverables including reports and statistical models
    \item hired a diverse team of analysts and data scientists
    \item trained the team in our process and procedures
    \item implemented modern technologies such as Posit Connect to provide
    consistent, reproducle results across a diverse range of data products
    \end{itemize}

    During my time with Acuitas, we integrated dozens of EHR systems into a
    vendor neutral EHR and payer format which enabled us to provide data
    products to clients regardless of their internal tech stack.
    
    Notable data products provided include:

    \begin{itemize}
      \item Federal HEDIS reports and tooling to identify opportunities to
      improve HEDIS scores
      \item Models and reports to accurately assess patient HCC scores
      \item Tools to monitor provider efficiency
      \item Models to identify high-risk patients such as patients at risk of
      falling
    \end{itemize}
  \end{position}

  Although I am proud of all the accomplishments listed above. The
  accomplishment I am most proud of was the culture of the team. The time/cost
  pressures placed on the Acuitas team is immense. Yet we were able to develop
  a culture of teamwork and trust which helped us support each other through a
  pandemic and successfully hit difficult timelines inside a broader climate
  which was not always supportive of what we were doing.

  %% PCG SSO -------------------------------------------------------------------
  \employer{\href{http://www.publicconsultinggroup.com/health/staffingsolutions/index.html}{PCG
      Staffing Solutions Organization - \\ New York State Department of Health}}
  \title{Healthcare Program Advisor}
  \dates{July 2017 - August 2017}
  \location{Albany, NY}
  \begin{position}
    This change in employer and job title are misleading. I moved from IPRO to
    PCG after PCG won the contract renewal. The state then instructed PCG to
    hire most of the IPRO staffers working on the contract to maintain
    continuity. I continued to report to the same DOH staff person and had the
    same responsibilities until I left in August.
  \end{position}

  %% IPRO ----------------------------------------------------------------------
  \employer{\href{http://ipro.org/}{IPRO (Island Peer Review Organization) - \\ New York State Department
      of Health}}
  \title{Healthcare Program Research Manager}
  \dates{March 2014 - June 2017}
  \location{Albany, NY}
  \begin{position}
    Embedded consultant with the New York State Department of
    Health. Supervise the evaluation of the State's diverse care
    management programs. Includes working with external program
    evaluators and internal stakeholders to understand the impact and
    effectiveness of care management programs and how to improve them.

    Responsibilities include care management programs administered by
    the Medicaid Managed Care Plans and Health Homes. Care management
    programs are an administrative (case management) intervention for
    people with complex health care needs and high utilization of
    emergency and inpatient services. Care management programs provide
    integrated, supportive, and preventative health care services
    intended to improve participants' health and well-being while
    reducing the unnecessary utilization of inpatient and emergency
    department services.

    Accomplishments:
    \begin{itemize}
    \item Monitor Emergency Department, Inpatient, and Primary Care
      utilization of Medicaid enrollees who agreed to receive care
      management services. Includes program improvement and pre-post
      analyses.
    \item Supervise the development and reporting of HEDIS quality
      metrics as listed in New York State's SPA to State Agency
      Partners, Health Homes and CMS.
    \item Redesigned the Care Management Assessment Reporting Tool:
      \href{https://www.health.ny.gov/health_care/medicaid/program/medicaid_health_homes/assessment_quality_measures/docs/hh_cmart_specs_v3.pdf}{HH
        CMART3 Specifications (PDF)}.
    \item Developing data-warehouse structure and ETL tools to combine
      care management data from a number of different data sources.
    \item Peer Leader - Epidemiology and Biostatistics Community of
      Practice (EBCoP). Developed the
      \href{http://choens.github.io/titanic/}{Titanic Introduction To
        R} workshops to introduce DOH research scientists to the R
      programming language. Manage other Learning Pathways workshops.
    \end{itemize}
  \end{position}

  % HZA --------------------------------------------------------------------------
  \employer{Hornby Zeller Associates}
  \title{Policy Analyst}
  \dates{September 2007 - March 2014}
  \location{Troy, NY}
  \begin{position}
    Hornby Zeller Associates (HZA) was a consulting firm specializing in child
    welfare, elder care, juvenile justice, and mental health. 
    
    As a policy analyst, I was responsible for developing research hypotheses
    and methodologies and providing support and mentoring to junior staff. Other
    responsibilities include data collection, management and analysis of
    research data and dissemination of results. I also initiated a transition
    from a workflow based on Access and SPSS to one based on MS SQL Server and R
    which improved the accuracy and timeliness of report completion and began
    the company's transition to reproducible reporting.

    Example projects:
    \begin{itemize}
    \item Program / Systems Evaluation: Arkansas (IV-E Waiver), West
      Virginia (Jacob's Law), Alaska (Mental Health Beneficiary Program)
    \item Time Study: Virginia, Westchester County, Wisconsin
    \item Accounting \& Title IV-E Funding: Wisconsin, Georgia, Mississippi
    \end{itemize}

    Accomplishments:
    \begin{itemize}
    \item Started an internship program and supervised two interns.
    \item Developed templates and processes to automate the production
      of reports.
    \item Provided staff training \& support for PostgreSQL, SQL
      Server, R, and SPSS.
    \item Mentored junior staff in of research methods and application
      of statistical tests.
    \item Introduced version control resulting in more reliable, reproducible results.
    \end{itemize}
  \end{position}

  % Coalition for the Homeless ---------------------------------------------------
  \employer{\href{www.coalitionforthehomeless.org}{Coalition for the Homeless}}
  \title{Principal Investigator}
  \dates{September 2006 - June 2007}
  \location{Albany, NY}
  \begin{position}
    Principal Investigator for an exploratory research project which
    established the existence of homeless men and women with
    borderline intellectual functioning in New York State homeless
    shelters.

    Accomplishments:
    \begin{itemize}
    \item Demonstrated the validity of the Mini Mental Status Exam to
      identify homeless adults with borderline intellectual functioning.
    \item Recruited, trained, and supervised a team of BSW students.
    \item Interviewed over 150 homeless New Yorkers. Successfully
      identified homeless adults with a historical diagnosis of
      borderline intellectual functioning.
    \item Presented results to the Assistant Commissioner
      of the Office of Mental Retardation and Developmental
      Disabilities\footnote{Now called the Office for People With
        Developmental Disabilities.}.
    \end{itemize}
  \end{position}


  \section{CLINICAL SOCIAL WORK EXPERIENCE:}

  % Catholic Charities -----------------------------------------------------------
  \employer{\href{http://www.ccrcda.org/}{Catholic Charities}}
  \title{Residential Habilitation Provider}
  \dates{October 2005 - September 2007}
  \location{Albany, NY}
  \begin{position}
    Delivered in-home services to Albany residents with mental
    retardation and other developmental disabilities. Services included
    family support, socialization skills, and crisis counseling.
  \end{position}

  % Joseph's House ---------------------------------------------------------------
  \employer{\href{http://www.josephshousetroy.org}{Joseph's House \& Shelter}}
  \title{Advocate / Case Manager}
  \dates{September 2005 - May 2006}
  \location{Troy, NY}
  \begin{position}
    Worked with homeless guests to find housing and income in the form
    of jobs or social services such as SSD. Addressed other needs on a case-by-case
    basis. Developed a Guest Satisfaction Survey for the shelter. Surveyed
    guests in the shelter during the winter of 2005 - 2006.
  \end{position}

  % Ramapo for Children ------------------------------------------------------------
  \employer{\href{http://www.ramapoforchildren.org/}{Ramapo for Children}}
  \title{Teen Leadership Program Director}
  \dates{June 2003 - September 2005}
  \location{Rhinebeck, NY}
  \begin{position}
    Supervised an outdoor experiential program for teens with
    developmental disabilities and mental health
    diagnoses. Responsibilities included staff supervision and program
    development. Implemented a unique peer support program for
    high-risk teens using conference calls. Led multi-day canoe trips
    on Indian Lake in the Adirondacks. Facilitated team building
    activities including low and high ropes facilitation.
  \end{position}

  \section{SOFTWARE DEVELOPMENT EXPERIENCE:}

  \begin{itemize}
    \item {\bf Data Analysis:} R, Python, Excel, SAS, SPSS
    \item {\bf Database:} PostgreSQL, MySQL, SQL Server, Vertica, Snowflake
    \item {\bf BI:} Posit Connect (SHINY), Tableau, Qlik, Excel 
    \item {\bf Reproducible Research/Reporting:} \LaTeX, Sweave, knitr, Org-Mode
    \item {\bf IT:} Linux server/workstation administration 
  \end{itemize}

  \section{RECENT PRESENTATIONS:}
  \begin{itemize}
    \item {\bf Health Analytics Summit (HAS):} {\href{https://www.youtube.com/watch?v=--vqwbJucPs}{Advanced Analytics for Medical Practices: Value-Based Care in the New Normal}}
    \item {\bf R In Epidemiology:} {\href{https://www.youtube.com/watch?v=-zhTXiiCj58}{Connecting Primary Care Providers to their own data}}
    \end{itemize}

    The above presentations are both available on Youtube and the
    links work in a digital version of this PDF.
  
  \section{EDUCATION:}
  \begin{tabular} {p{2.25in} p{2.75in} p{2in}}
    Master of Social Work & The State University of New  York at Albany & Albany, NY 2007 \\
    Bachelor of International Affairs & Georgia Institute of Technology & Atlanta, GA 2002 \\
  \end{tabular}

  % \vspace{.125in}

\end{resume}

\vspace{.25in}
Updated: \today \\
\LaTeX code for this document available at:
\href{https://github.com/Choens/Resume/}{https://github.com/Choens/Resume/}


\end{document}
